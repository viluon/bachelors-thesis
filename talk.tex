% arara: xelatex
% arara: makeglossaries
% arara: xelatex
% arara: xelatex

% style inspired by
% https://github.com/omelkonian/presentations/blob/2c0b2e1f592a8f90797e4997c3ab8785b114f595/%5B2019.08.20%5D%20BitML%20(SRC%20Presentation%20@%20ICFP)/bitml-presentation.tex

\documentclass[aspectratio=169]{beamer}
\usetheme{metropolis}

\usepackage[utf8]{inputenc}
\usepackage[T1]{fontenc}
\usepackage{textcomp}
\usepackage[english]{babel}
\usepackage{amsmath, amssymb, bm}
\usepackage{tcolorbox}
\usepackage[makeroom]{cancel}
\usepackage{mathpartir}
\usepackage{minted}

% figure support
\usepackage{import}
\usepackage{xifthen}
\usepackage{pdfpages}
\usepackage{transparent}
\newcommand{\incfig}[1]{%
	\def\svgwidth{\columnwidth}
	\import{./figures/}{#1.pdf_tex}
}

% style for tcolorboxes
\tcbset{plain/.style={colbacktitle=white,coltitle=black,colback=white}}
\pdfsuppresswarningpagegroup=1

% fonts
\usepackage{relsize}
\usepackage[tt=false]{libertine}
\usepackage[libertine]{newtxmath}

% colours
\definecolor{CTUBlue}{HTML}{007ac2}
\setbeamercolor{alerted text}{fg=CTUBlue}


\title{Haskell Dynamic Tracing}
\author{Ondřej Kvapil}
\institute{Faculty of information technology, Czech technical university in Prague}
\date{28th April, 2021}


\begin{document}

\begin{center}
	\maketitle
\end{center}

\begin{frame}{Outline}
	\begin{itemize}
		\item Introduction
		\item Goals of the thesis
		\item Motivation behind the choice of topic
		\item Technical terminology and acronyms
		\item Current state of the solution
		\item Solution of the problem (expected outcome)
		\item Conclusion -- a summary of the important points, contributions
		\item Thanks \& discussion
	\end{itemize}
\end{frame}

\begin{frame}{Introduction}
	\begin{itemize}
		\item Haskell is a \alert{lazy} language \pause
		\item Delays evaluation of an expression until its value is needed
	\end{itemize}
	\begin{minted}{haskell}
		foo
	\end{minted}
\end{frame}

\begin{frame}{Goals}
	\begin{itemize}
		\item Key question: how is laziness used in practice?
	\end{itemize}
\end{frame}

\end{document}
