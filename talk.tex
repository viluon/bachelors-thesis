% arara: xelatex
% arara: makeglossaries
% arara: xelatex
% arara: xelatex

% style inspired by
% https://github.com/omelkonian/presentations/blob/2c0b2e1f592a8f90797e4997c3ab8785b114f595/%5B2019.08.20%5D%20BitML%20(SRC%20Presentation%20@%20ICFP)/bitml-presentation.tex

\documentclass[aspectratio=169]{beamer}
\usetheme{metropolis}

\usepackage[utf8]{inputenc}
\usepackage[T1]{fontenc}
\usepackage{textcomp}
\usepackage[english]{babel}
\usepackage{amsmath, amssymb, bm}
\usepackage{tcolorbox}
\usepackage[makeroom]{cancel}
\usepackage{mathpartir}
\usepackage{xparse,minted}
\usepackage{csvsimple}

% figure support
\usepackage{import}
\usepackage{xifthen}
\usepackage{pdfpages}
\usepackage{transparent}
\newcommand{\incfig}[1]{%
	\def\svgwidth{\columnwidth}
	\import{./figures/}{#1.pdf_tex}
}

% style for tcolorboxes
\tcbset{plain/.style={colbacktitle=white,coltitle=black,colback=white}}
\pdfsuppresswarningpagegroup=1

% fonts
\usepackage{relsize}
\usepackage[tt=false]{libertine}
\usepackage[libertine]{newtxmath}

% colours
\definecolor{CTUBlue}{HTML}{007ac2}
\setbeamercolor{alerted text}{fg=CTUBlue}


\title{Haskell Dynamic Tracing}
\author{Ondřej Kvapil, supervised by Ing. Filip Křikava, PhD.}
\institute{Czech Technical University in Prague, Faculty of Information Technology}
\date{28th April, 2021}


\begin{document}

\begin{center}
	\maketitle
\end{center}

%\begin{frame}{Outline}
%	\begin{itemize}
%		\item Introduction
%		\item Goals of the thesis
%		\item Motivation behind the choice of topic
%		\item Technical terminology and acronyms
%		\item Current state of the solution
%		\item Solution of the problem (expected outcome)
%		\item Conclusion -- a summary of the important points, contributions
%		\item Thanks \& discussion
%	\end{itemize}
%\end{frame}

\begin{frame}[fragile]{Introduction}
	\begin{itemize}
		\item<1-> Haskell is a \alert{lazy} language
		\item<2-> Delays evaluation of an expression until its value is needed
		\item<6-> Shares evaluated subexpressions
	\end{itemize}
	\begin{overprint}
		\onslide<3-5>
		\begin{minted}[autogobble]{haskell}
			snd :: (Int, Int) -> Int
			snd (x, y) = y
		\end{minted}
		\begin{overprint}
			\onslide<4>
			\begin{minted}[autogobble]{haskell}
				foo :: Int
				foo = snd (complexComputation, 3)
			\end{minted}
			\onslide<5>
			\begin{minted}[autogobble]{haskell}
				foo' :: Int
				foo' = snd (error "oops!", 3)
			\end{minted}
		\end{overprint}
		\onslide<7>
		\begin{minted}[autogobble]{haskell}
			let x = complexComputation
			in  f (x, x)
		\end{minted}
	\end{overprint}
\end{frame}

\begin{frame}{Goals}
	\begin{itemize}
		\item Key question: how is laziness used in practice? \pause
		\item Develop a \alert{dynamic tracing} framework \pause
		\item Tool capable of analysing real-world Haskell programs
	\end{itemize}
\end{frame}

\begin{frame}{Motivation}
	\begin{itemize}
		\item Interesting landscape -- Haskell is heavily used in academia
			\pause
		\item Lack of empirical results despite extensive theoretical study
			\pause
		\item Laziness has advantages as well as significant downsides
	\end{itemize}
\end{frame}

\begin{frame}{State-of-the-art}
	\begin{itemize}
		\item Unnecessary laziness leads to \alert{large memory overhead}
			\pause
		\item Some tools designed to avoid too much laziness
			(\texttt{nothunks}, \texttt{BangPatterns}) \pause
		\item Black-box problem: developers lack insight about the runtime
			state
	\end{itemize}
\end{frame}

\begin{frame}{Our solution}
	\begin{itemize}
		\item Compiler plugin for the \alert{Glasgow Haskell Compiler} \pause
		\item Transforms surface syntax of Haskell to add tracing calls \pause
		\item Traces evaluation of function calls and function arguments
	\end{itemize}
\end{frame}

\begin{frame}[fragile]{Our solution}
	\begin{overprint}
		\onslide<1>
		\begin{minted}[autogobble]{haskell}
			qsort [] = []
			qsort (a:as) = qsort left ++ [a] ++ qsort right
				where (left, right) = (filter (<=a) as, filter (>a) as)

			main = print $ qsort [1 + 8, 4, 0, 3, 1, 23, 11, 18]
		\end{minted}

		\onslide<2>
		\begin{minted}[autogobble]{haskell}
			qsort (a : as) = let !call_number_1 = traceEntry "qsort"
			                 in qsort left
			                       ++ [(traceArg "qsort") "a" call_number_1 a]
			                       ++ qsort right
			  where
			  (left, right) = let !call_number_2 = traceEntry "qsort"
			                   in (filter
			                         (<= (traceArg "qsort") "a" call_number_2 a)
			                         (traceArg "qsort") "as" call_number_2 as,
			                       filter
			                         (> (traceArg "qsort") "a" call_number_2 a)
			                         (traceArg "qsort") "as" call_number_2 as)
		\end{minted}

	\end{overprint}
\end{frame}

\begin{frame}{Our solution}
	\csvautotabular{trace-slice-with-header.csv}
\end{frame}

\begin{frame}{Summary}
	\begin{itemize}
		\item Compile-time \alert{rewriting of programs} via a plugin for
			the Glasgow Haskell Compiler \pause
		\item Annotation with side-effecting tracing functions \pause
		\item Compiled program records a trace of relevant events alongside
			regular output
	\end{itemize}
\end{frame}

\begin{frame}{}
	Thank you for listening.

	\bibliographystyle{iso690.bst}
	\begin{thebibliography}{1}
		\bibitem{thesis}
		Kvapil, Ondřej.
		\textit{Haskell Dynamic Tracing}.
		Bachelor's thesis.
		Czech Technical University in Prague,
		Faculty of Information Technology, 2021.
	\end{thebibliography}
\end{frame}

\end{document}
